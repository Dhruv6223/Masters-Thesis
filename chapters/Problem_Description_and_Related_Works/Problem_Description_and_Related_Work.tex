\chapter{Background\_V1 \label{Chapter_Background}}

\large  The internet and computers have played significant role in a transformation of documents. Before documents got digitized, it used to be in form of printed papers. Most of the documents have their own layouts which is highly dependent on the language since the language comes with its own grammatical rules and the way of sentence formation. Once the smartphone came into the world, it started influencing the way of documents being prepared. To understand the text document has always been a part of interest in computer science. Since the impact of having a system that can understand the text document correctly, its repeatable functionality and speed that could help industries or companies to bring down the cost and time they are spending on dealing with documents on their day to day life. However, there are many steps in between to address before machine can read and understand text and layout information in text documents. Start with the letters, words and sentences. These are the basic and prime components of any language. One can not pass any information without it. One can argue about passing information in terms of symbol or a figure but at the basic level a letter is like a symbol or a figure that represents some value or information. We as a human being can easily recognise the texture and the layout of the letter by just looking at it, which is not the case for computers. Till the date computers are using 1s and 0s to perform tasks. Therefore it is a challenge to be able to recognise and process the letters that have been used in word or sentence while generating the document and form a structure of 1s and 0s so that machine can understand. 

Compilers are specifically design to understand high-level programming languages and translate it to the machine code, but when it comes to read a document, it is difficult for them since document does not contains the rules similar to programming languages and in addition, every text documents have the basic blocks of any human language, "letters". Character recognition techniques refers to a symbolic identity of a character within symbol or a character. This includes the replication/recognition of human functions like machine printed and hand printed/cursive-written characters by machines. Character recognition is known as optical character recognition since it deals with optically processed characters. The first mention of OCR was for as an aid to the visually handicapped and the first successful attempt was achieved by Russian Scientist Tyurin in 1900 \cite{govindan1990character}. Later on, the application of OCR changed into data processing systems in business world since it can deal with the enormous flood of paper documents such as bank cheques, commercial forms, credit card imprints, governmental records, mails and so on. Before smartphones and digital camera, documents were simply printed papers, However since smartphones and digital cameras got introduced, the number of documents increased since documents can now be easily produced digitally. Though generating documents got digitized, not all the documents are in machine readable or processable format. Most of them are just images or scans. Where OCR allowed to deal with these digitally born document such as scanned paper-documents, PDF files or images captured by a digital camera into machine readable, editable and searchable data. In paper \citep{AnOverviewoftheTesseractOCREngine} author have described in detail functionality of methods of famous Tesseract OCR Engine. Author talk about the development of open-source OCR engine "Tesseract" that it was developed at HP between 1984 and 1994. In the infancy era for Tesseract, the accuracy was a challenge, and it was major PhD research project in HP Labs Bristol. It was 1994, HP presented the final developed OCR engine to UNLV for the 1995 Annual Test of OCR Accuracy\cite{UNLV_4th_annual_test_ocr}, where it got proved that it was worth against the commercial engines that exists that time. Later in 2005, HP made Tesseract for open source which is available at \url{http://code.google.com/p/tesseract-ocr}. According to \cite{AnOverviewoftheTesseractOCREngine}, the functionality of Tesseract OCR will be described in CHAPTER METHODS.


OCR provides ability to extract the letters out of documents and compilers can translate the specific language structure in a way that machine can understand, However, there is some mechanism missing between compiling High-language programming language and understanding the letters, words or sentences derived from parsing the document. Language plays a dominant role in terms of passing information, thoughts and ideas. In addition, different regions, countries and part of the world have different languages that varies in structure, grammar, letters and so on making it difficult to come with the logic or mathematical rule which can apply on one or more languages in order to understand for machines. The early approach of Noam Chomsky \cite{robert1957review}, who introduced syntactic structures as a formalized theory of linguistic structure. He introduced rules based on universal grammar: However languages are generally not well-defined and lack in stable structure hence the approach had several drawbacks. Although, the approach was not perfect, it started the true evolution in NLP. Later on the trend shifted to using probabilities and statistics for machine translation but the progress was slower than expected resulting in less funding. It was the year 1969, Roger Schank presented the concept of using tokens \cite{tokenization_history} which is still being used in NLP, Tokens provided better grasp to map sentences, it can provide better insights to machine with detailed information at object-level. 

\input{chapters/Problem_Description_and_Related_Works/Document_understanding_for_V1}

\section{Machine Learning and Deployment \label{deployment_section}}
\acrshort{ml} application has developed from being in domain of academic research to an applied field. A survey conducted by McKinsey \& Company \cite{analytics2019global} shows that nearly 25\% of business processes are adopting machine learning techniques. However, there is a huge difference and challenges to put ML in real world system as compared to academic settings. Reports from Algorithmia \cite{wiggers2019algorithmia, hecht2019add} shows that it takes 8 to 90 days for the majority of companies to deploy a single model and 18\% of companies took even more time to deploy. According to \acrfull{idc}'s report\footnote{\url{https://venturebeat.com/ai/idc-for-1-in-4-companies-half-of-all-ai-projects-fail/},Accessed: 22.03.2024} that includes 2,473 organizations into the survey, A notable portion has failed in an attempt of \acrshort{ai} deployments. Moreover, the \acrshort{idc}'s report points that the reasons could be lack of expertise, bias in data and high costs of resources.

When we talk about deploying ML functionality in production, there are several aspects that should be discuss such as machine learning deployment workflow, ethical considerations, law, end-user's trust, and security which are briefly discussed in \cite{paleyes2022challenges}. The term to make a service or a product using machine learning and making available for users refer as production. Sometimes the term ML deployment workflow also known as ML pipelines and there are various definitions and descriptions of it such as Cros-Industry Standard Process For Data Mining (CRISP-DM) \cite{shearer2000crisp} or Team Data Science Process (TDSP) \cite{TDSP}. In general, the process of developing machine learning based product and services in industrial environment have stages like data management, model learning, model verification and model deployment. Each of these stages can be further broken down in smaller steps as shown in \Cref{tab:deployment_stages} and each steps can run in parallel while informing each other through feedback loops. For developing ML pipelines, one could think of using best practices to use software development principles for productionizing machine learning products and services such as DevOps. It is about fast, flexible and provisioning business processes that integrates development, delivery and operations more efficiently thus pronounce as DevOps. It was an organizational shift from distributed groups or departments that performs function separately to cross-functional teams that works on continuous operational feature deliveries. These principles of DevOps brought a cultural shift towards in the collaboration between areas like development, quality assurance and operations. In paper \cite{ebert2016devops}, authors have presented a comprehensive case study of different tools and micro-services and the impact of these DevOps technologies on industry projects, An overall DevOps model's areas are shown in \Cref{fig:DevOps}. 

\begin{figure}[!ht]
    \centering
    \includegraphics[width=0.45 \textwidth]{chapters/images/Literature_review/DevOps.JPG}
    \caption{A DevOps model infrastructure \cite{ebert2016devops}}
    \label{fig:DevOps}
\end{figure}

Some of these DevOps principles can be directly apply to ML systems but there are number of challenges that are specific to the machine learning, which are briefly discussed in \cite{dang2019aiops}, this paper introduced the term AIOps (also recognized as MLOps) that refer as DevOps tasks for ML systems. The market for machine learning services and tools is already started gaining growth. There are new tools and services being introduced continuously in order to overcome the problems in the deployment process. For example platforms like AWS SageMaker\footnote{\url{https://aws.amazon.com/sagemaker/},Accessed: 22.03.2024}, AzureML\footnote{\url{ https://azure.microsoft.com/en-us/products/machine-learning}, Accessed: 22.03.2024}, TensorFlow TFX\footnote{\url{https://www.tensorflow.org/tfx}, Accessed: 22.03.2024}, MLflow\footnote{\url{https://mlflow.org/}, Accessed: 22.03.2024} and so on helps in various stages of deployment by providing services like data storage, retraining and model hosting with \acrfull{api} for training and inference operations, special set of metrics for monitoring model performance and health and interface for custom changes. These platforms offer managed infrastructures that helps decreasing the burden on the people associated to maintain the operations of ML model in production. Platforms like these allowed people to actively contribute in a communities and build tools and libraries for different aspects in deployment stages. For example, to check quality, CheckList methodology \cite{ribeiro2020beyond} gives formal approach to check the quality of \acrshort{nlp} models, The Data Linter \cite{hynes2017data} to inspect the dataset for potential issues. Tools like Auto-keras\footnote{\url{https://autokeras.com/}, Accessed:03.04.2024}, Auto-sklearn\footnote{\url{https://www.automl.org/automl-for-x/tabular-data/auto-sklearn/}, Accessed: 03.04.2024} aims to provide general-purpose implementations for machine learning algorithms. Though new tools for ML tasks are being released constantly, Practitioner still have to have knowledge of right tool and the dependencies at different deployment stages.




%######################devOps bib name : ebert2016devops


% Please add the following required packages to your document preamble:
% \usepackage{multirow}
% Please add the following required packages to your document preamble:
% \usepackage{multirow}
% Please add the following required packages to your document preamble:
% \usepackage{multirow}
% Please add the following required packages to your document preamble:
% \usepackage{multirow}
\begin{table}[H]
\begin{tabular}{|l|l|l|}
\hline
                  \textbf{Deployment Stage}&\textbf{Deployment Step}& \textbf{Considerations, Issues, and Concerns} \\ \hline
\multirow{7}{*}{Data management} &                   Data collection&  Data discovery\\ \cline{2-3} 
                  & \multirow{2}{*}{Data preprocessing} &  Data dispersion\\  
                  &                                     &  Data cleaning\\ \cline{2-3} 
                  & \multirow{3}{*}{Data augmentation}  &  Labeling of large volumes of data\\ 
                  &                                     &  Access to experts\\ 
                  &                                     &  Lack of high-variance data\\ \cline{2-3} 
                  &                   Data analysis     &  Data profiling\\ \hline
\multirow{9}{*}{Model learning} & \multirow{2}{*}{Model selection}   &  Model complexity\\ 
                  &                                     &  Resource-constrained environments\\ 
                  &                                     &  Interpretability of the model\\ \cline{2-3} 
                  & \multirow{3}{*}{Training}           &  Computational cost\\ 
                  &                                     &  Environmental impact\\ 
                  &                                     &  Privacy-aware training\\ \cline{2-3} 
                  & \multirow{3}{*}{Hyper-parameter selection} &  Resource-heavy techniques\\ 
                  &                                     &  Unknown search space\\ 
                  &                                     &  Hardware-aware optimization\\ \hline
\multirow{6}{*}{Model verification} & \multirow{2}{*}{Requirement encoding} & Performance metrics \\
                  &                                     & Business-driven metrics\\ \cline{2-3} 
                  &                 Formal verification &  Regulatory frameworks\\ \cline{2-3} 
                  & \multirow{3}{*}{Test-based verification} &  Simulation-based testing\\ 
                  &                                     &  Data validation routines\\ 
                  &                                     &  Edge case testing\\ \hline
\multirow{9}{*}{Model deployment} & \multirow{4}{*}{Integration} &  Operational support\\ 
                  &                   &  Reuse of code and models\\ 
                  &                   &  Software engineering anti-patterns\\ 
                  &                   &  Mixed team dynamics\\ \cline{2-3} 
                  & \multirow{3}{*}{Monitoring} & Feedback loops  \\ 
                  &                   &  Outlier detection\\
                  &                   &  Custom design tooling\\ \cline{2-3} 
                  & \multirow{2}{*}{Updating} & Concept drift  \\
                  &                   &  Continuous delivery\\ \hline
\end{tabular}
\caption{Considerations, Issues and Concerns in different deployment stage \cite{paleyes2022challenges}}
\label{tab:deployment_stages}
\end{table}

































\section{Cloud Computing for ML Deployment}

Internet keeps changing the way people work, learn, communicate and so on. It has influenced from one individual to entire industries. Rapid development of processing and storage technologies helped to reduce the cost of computing while increasing power and availability. This technological advancement provided a realization of a computing model called "Cloud Computing". Users can lease and release the resources like CPU and Storage in an on-demand manner. In general, the cloud computing infrastructure can be divided into two prime roles. One, the infrastructure provider responsible to manage cloud platforms. Second, service-provider, who consume these resources from infrastructure-providers and servers different services to end users. Cloud technologies have influenced the Information Technology (IT) industries and Companies like Google, Microsoft and Amazon provides cloud-platforms which help enterprises to develop, reshape their business models and gain benefits such as no up-front investment in resources since now they can just rent the infrastructure as they uses, which is also known as pay-as-you-go pricing model, high scalability, easy access and more important risks and maintenance. Using cloud infrastructure means we are simply outsourcing the risks such as hardware failures to the infrastructure providers who are better equipped to manage these risks that helps to decreasing the cost on maintenance and training of staff. In paper \cite{lee2013view}, Author presents results of an economic view of IBM cloud computing engagements \ref{tab:IT_benifits}.


\begin{table}[hb]
    \centering
    \begin{tabular}{|c|c|c|c|}
    \hline
         & Tasks & Traditional Computing &  Cloud Computing \\
    \hline
    \multirow{6}{5em}{Increasing speed and flexibility} & Test provisioning & Weeks & Minutes \\ 
        & Change Management & Months & Days/hours \\ 
        & Release Management & Weeks & Minutes \\
        & Service Access & Administered & Self-service \\ 
        & Standardization & Complex & Reuse/Share \\
        & Metering/billing & Fixed Cost & Variable Cost \\
    \hline
    \multirow{2}{5em}{Reducing Costs} & Server/storage utilization & 10-20\% & 70-90\% \\
        & Payback period & Years & Months \\
    \hline
    \end{tabular}
    \caption{Benifits of Cloud Computing}
    \label{tab:IT_benifits}
\end{table}



\subsection{Overview of cloud Architecture}

The architecture of cloud computing environment is made of sublayers such as Infrastructure as a service (Iaas),  Platform as a Service (Paas) and Software as a Service (SaaS) as show in \Cref{fig:Layers_of_Cloud_Architecture}. Each layer is coupled with layers above or below in a way that each layer can evolved separately allowing applications to be better at management and maintenance. 

\begin{figure}[!ht]
    \centering
    \includegraphics[width=0.5\textwidth]{chapters/images/Cloud_Computing/Cloud_layers.png}
    \caption{Layers of Cloud Architecture}
    \label{fig:Layers_of_Cloud_Architecture}
\end{figure}

\subsubsection{Infrastructure as a Service - IaaS}
The layer IaaS includes resources like data centers that contains servers, routers, power and cooling systems and so on. Therefore, also known as a hardware layer. It also contains the infrastructure layer that includes pool of storage and computing resources like virtualization technologies therefore also known as virtualization layer. IaaS usually refers to providing resources on-demand mostly in forms of Virtual Machines for instance Amazon EC2\footnote{\url{https://aws.amazon.com/pm/ec2/}, Accessed: 06.05.2024}.

\subsubsection{Platform as a Service - PaaS}

This layer is build on top of the infrastructure layer that includes different operating systems and application frameworks. PaaS-provider delivers necessary hardware and software tools over internet for users which allows to focus on deployment and management of their application. 

\subsubsection{Software as a Service - SaaS}

Saas is the layer where the actual cloud application is being served that reaches to end-user and it is at the top of the hierarchy. It is different from ordinary served application in terms of highly-scalable since this layers have automatic-scaling feature to achieve better performance, availability and lower operating cost.

\subsection{Machine Learning and Cloud Computing}

Cloud infrastructure outperforms the traditional way of serving the ML models in terms of performance and cost, many of these cloud provider are focusing on developing new architectures that are specifically developed for workloads like neural networks and deep learning, Implementing features like specialized cores to increase matrix operations for instance Google TPUs (Tensor Processing Units) \cite{google_tpu}. In addition, the elasticity, flexibility and economy is making cloud computing more suitable for deploying machine learning models. The different layer model of cloud makes it easy to deploy various infrastructure components on different layers that helps to manage these resources in efficient ways. Cloud providers offers inbuilt PaaS and SaaS a step ahead from low-level offerings like IaaS to enable large-scale computing infrastructures with easy-to-use services with respect to Machine learning "as-a-Service" for end users. Today's data-centers are located in various part of the world. Using services available in PaaS/ SaaS one can deploy their services or product on distributed infrastructures which is accessible trough internet. 

Cloud computing enabled companies to build their products and collaborate on a global level. For instance Hugging face \cite{huggingfacehub} provides a platform where companies or individuals can build their own AI, leverage open source models and technology and make it easy for data scientist, machine learning engineers and developers to collaborate. Users can easily access these models and applications with the hardware capabilities of Google Cloud \cite{Googlecloud} such as TPU instances, Virtual Machines, NVIDIA H100 Tensor Core GPUs \cite{Nvidiagpu} and so on. Currently, there are approx more than 400 thousand models, 150 thousand application and 100 thousand datasets available on hugging face \cite{huggingfacehub}. Some of them may come with copyrights or Licences to use and assets like  Open-Source, anyone can access to these pre-trained models which are already well trained and architected, one can fine-tune it and use it for their downstream tasks with their own datasets or the available datasets without investing on large resource required to set up IT infrastructure to serve high computing services like machine learning on Hugging face by just using internet through their laptops in a pay-as-you-go manner.






